\documentclass{classrep}
\usepackage[utf8]{inputenc}
\usepackage{color}
\usepackage{amsmath}

\studycycle{Informatyka, studia STACJONARNE, I st.}
\coursesemester{VI}

\coursename{Komputerowe systemy rozpoznawania}
\courseyear{2020/2021}

\courseteacher{prof. dr hab. inż. Adam Niewiadomski}
\coursegroup{poniedziałek, 12:00}

\author{
  \studentinfo{Hubert Gawłowski}{224298} \and
  \studentinfo{Kamil Kiszko-Zgierski}{224328} }

\title{Projekt 1. Klasyfikacja dokumentów tekstowych}

\begin{document}
\maketitle

\section{Cel projektu}

Celem zadania jest zaimplementowanie algorytmu $k$-NN w technologii Java na potrzeby klasyfikacji tekstów 
oraz zbadanie wpływu wybranych cech liczbowych i tekstowych na skuteczność powyższej metody. 
Badanie zostanie przeprowadzone na podstawie artykułów prasowych z agencji prasowej Reuters.  \\



\section{Klasyfikacja nadzorowana metodą $k$-NN}
Algorytm $k$-NN (od angielskich słów nearest neighbour - najbliższy sąsiad) to algorytm, 
którego działanie polega na przyporządkowaniu obiektu poddanego rozpoznawaniu do jednej z klas.
 Do wykorzystania tego algorytmu niezbędny jest zestaw klas, do których może należeć obiekt, zbiór danych uczących oraz rozpoznawany obiekt.
Metoda $k$-NN należy do grupy metod minimalnoodległościowych, ponieważ o zaklasyfikowaniu obiektu 
do danej klasy decyduje najmniejsza odległość (zgodna z przyjętą metryką), pomiędzy rozpoznawanym obiektem oraz k-obiektami z ciągu uczącego. 
Wyszukiwanie najmniejszej odległości pomiędzy obiektami można przedstawić za pomocą wzoru ogólnego:
\begin{gather}
\rho(x, x ^ {i, k})= \min_{x^\mu \in U^i}(x, x^\mu)
\end{gather}
\indent gdzie $\rho$ to wybrana metryka, U$^$i oznacza ciąg uczący, $x^{i, k}$ jest elementem zbioru U$^$i, a rozpoznawany obiekt to x \cite{tadeusiewicz90}.\\
\indent Skuteczność algorytmu $k$-NN  mierzona jest na podstawie odsetka poprawnych przyporządkowań obiektów do odpowiadających im klas. 


\subsection{Ekstrakcja cech, wektory cech}

Pierwszym etapem, który należy wykonać w procesie rozpoznawania tekstów jest wyodrębnienie takich cech, 
aby jak najlepiej określały ich charakterystykę.Wszystkie artykuły są napisane w tym samym języku oraz w tej samej formie stylistycznej, 
dlatego w trakcie analizy skupiliśmy się na cechach liczbowych oraz tekstowych. Mając to na uwadzę, dokonaliśmy ekstrakcji poniższych cech:
\begin{enumerate}
    \item Zapis cyfr - za pomocą ciągu cyfr otrzymamy informacje o np. numerach telefonu, które to są charakterystyczne w zależności od kraju.
    \item Waluty - wyciągnięcie z tekstów nazw najczęsciej używanych walut. Kraje, z których pochodzą porównywane teksty używają różnych walut.
    \item Częstość występowania dat - zliczenie, jak często w podanych tekstach występują daty. Zakładamy, że w zależności od tego, z którego kraju dotyczy tekst, częstość wystąpień zapisów datowych będzie się różnić.
    \item Format zapisu dat - w zależności od kraju format zapisu dat różni się.
    \item Ogólna liczba słów - zliczenie wszystkich słów występujących w tekście. Uważamy, że w zależności od tego, jakiego kraju tekst dotyczy, ich długość może być różna
    \item Częstość słów rozpoczynających się wielką literą - słowa takie będą oznaczały najczęściej  nazwy własne np. imiona, nazwiska, nazwy budynków. Pisząc o jednym kraju może być używane więcej takich słów, 
	a o innych mniej. Z tej grupy wykluczymy jednak wyrazy składające się wyłącznie z wielkich liter, o których mowa będzie w punkcie następnym.
    \item Częstość słów pisanych wielką literą - najczęściej będą to skróty. Uważamy, że w zależności od opisywanego kraju, ilość wykorzystywanych skrótów może się różnić.
    \item Układ SI/imperialny - zdecydowanie w krajach anglojęzycznych częściej w tekstach stosowany będzie układ imperialny, natomiast w pozostałych - układ SI.
    \item Częstość występowania cytatów - uważamy, że występuje wyraźna różnica w liczbie wykorzystanych cytatów, w zależności od opisywanego kraju.
    \item Słowa kluczowe - sporządzona zostanie lista słów kluczowych (np. elementy lokalizacyjne) dla krajów, które bierzemy pod uwagę w procesie klasyfikacji 
    \item Najczęściej występujące słowa - znalezienie słów, które najczęściej występują w tekstach o danym kraju
\end{enumerate}

Wektor wyekstrahowanych cech będzie się prezentował następująco: 
\begin{gather}
v = [c_{1}, c_{2}, c_{3}, c_{4}, c_{5}, c_{6}, c_{7}, c_{8}, c_{9}, c_{10}, c_{11}]
\end{gather}

\subsection{Miary jakości klasyfikacji} 
Miary jakości klasyfikacji (Accuracy, Precision,
Recall, F1). We wprowadzeniu zaprezentować minimum teorii potrzebnej do realizacji
zadania, tak by inżynier innej specjalności zrozumiał dalszy opis.\\
\indent Stosowane wzory, oznaczenia z objaśnieniami znaczenia symboli użytych w
doświadczeniu. Oznaczenia jednolite w obrębie całego sprawozdania.  Opis zawiera przypisy do bibliografii zgodnie z
Polską Normą, (zob. materiały BG PŁ).\\
\noindent {\bf Sekcja uzupełniona jako efekt zadania Tydzień 03 wg Harmonogramu Zajęć
na WIKAMP KSR.}


\section{Klasyfikacja z użyciem metryk i miar podobieństwa tekstów}
Wzory, znaczenia i opisy symboli zastosowanych metryk z
przykładami. Wzory, opisy i znaczenia miar
podobieństwa tekstów zastosowanych w obliczaniu metryk dla wektorów cech z
przykładami dla każdej miary \cite{niewiadomski08}.  Oznaczenia jednolite w obrębie całego sprawozdania.  Wstępne wyniki miary Accuracy dla próbnych klasyfikacji na ograniczonym zbiorze tekstów (podać parametry i kryteria
wyboru wg punktów 3.-8. z opisu Projektu 1.). \\ 
\noindent {\bf Sekcja uzupełniona jako efekt zadania Tydzień 04 wg Harmonogramu Zajęć
na WIKAMP KSR.}

\section{Budowa aplikacji}
\subsection{Diagramy UML}
Diagramy UML i zwięzłe opisy: idei aplikacji, modułu ekstrakcji i modułu
klasyfikatora.\\
\noindent {\bf Sekcja uzupełniona jako efekt zadania Tydzień 03 wg Harmonogramu Zajęć
na WIKAMP KSR.}

\subsection{Prezentacja wyników, interfejs użytkownika} 
Krótki ilustrowany opis jak użytkownik może korzystać z aplikacji, w~szczególności wprowadzać parametry klasyfikacji i odczytywać wyniki. Wersja JRE i inne wymogi
niezbędne do uruchomienia aplikacji przez użytkownika na własnym komputerze. \\
\noindent {\bf Sekcja uzupełniona jako efekt zadania Tydzień 04 wg Harmonogramu Zajęć
na WIKAMP KSR.}

\section{Wyniki klasyfikacji dla różnych parametrów wejściowych}
Wyniki kolejnych eksperymentów wg punktów 2.-8. opisu projektu 1.  Wykresy i tabele
obowiązkowe, dokładnie opisane w ,,captions'' (tytułach), konieczny opis osi i
jednostek wykresów oraz kolumn i wierszy tabel.\\ 

{**Ewentualne wyniki realizacji punktu 9. opisu Projektu 1., czyli,,na ocenę 5.0'' i ich porównanie do wyników z
części obowiązkowej**.}\\

\noindent {\bf Sekcja uzupełniona jako efekt zadania Tydzień 05 wg Harmonogramu Zajęć
na WIKAMP KSR.}


\section{Dyskusja, wnioski}
Dokładne interpretacje uzyskanych wyników w zależności od parametrów klasyfikacji
opisanych w punktach 3.-8 opisu Projektu 1. 
Szczególnie istotne są wnioski o charakterze uniwersalnym, istotne dla podobnych zadań. 
Omówić i wyjaśnić napotkane problemy (jeśli były). Każdy wniosek/problem powinien mieć poparcie
w przeprowadzonych eksperymentach (odwołania do konkretnych wyników: wykresów,
tabel). \\
\underline{Dla końcowej oceny jest to najważniejsza sekcja} sprawozdania, gdyż prezentuje poziom
zrozumienia rozwiązywanego problemu.\\

** Możliwości kontynuacji prac w obszarze systemów rozpoznawania, zwłaszcza w kontekście pracy inżynierskiej,
magisterskiej, naukowej, itp. **\\

\noindent {\bf Sekcja uzupełniona jako efekt zadania Tydzień 06 wg Harmonogramu Zajęć
na WIKAMP KSR.}


\section{Braki w realizacji projektu 1.}
Wymienić wg opisu Projektu 1. wszystkie niezrealizowane obowiązkowe elementy projektu, ewentualnie
podać merytoryczne (ale nie czasowe) przyczyny tych braków. 


\begin{thebibliography}{0}
\bibitem{tadeusiewicz90} R. Tadeusiewicz: Rozpoznawanie obrazów, PWN, Warszawa, 1991.  
\bibitem{niewiadomski08} A. Niewiadomski, Methods for the Linguistic Summarization of Data: Applications of Fuzzy Sets and Their Extensions, Akademicka Oficyna Wydawnicza EXIT, Warszawa, 2008.
\end{thebibliography}

Literatura zawiera wyłącznie źródła recenzowane i/lub o potwierdzonej wiarygodności,
możliwe do weryfikacji i cytowane w sprawozdaniu. 
\end{document}
