\documentclass{classrep}
\usepackage[utf8]{inputenc}
\usepackage{color}

\studycycle{Informatyka, studia STACJONARNE, I st.}
\coursesemester{VI}

\coursename{Komputerowe systemy rozpoznawania}
\courseyear{2020/2021}

\courseteacher{tytuł/stopień imię nazwisko}
\coursegroup{dzień, godzina zajęć}

\author{
  \studentinfo{Autor Pierwszy}{nr albumu 1} \and
  \studentinfo{Autor Drugi}{nr albumu 2} }

\title{Projekt 2.  Podsumowania lingwistyczne relacyjnych baz danych}

\begin{document}
\maketitle

Opis projektu ma formę artykułu naukowego lub raportu z zadania
badawczego/doświadczalnego/obliczeniowego (wg indywidualnych potrzeb związanych np. z
pracą inżynierską/naukową/zawodową). \\
\indent {\bf Wybrane sekcje (rozdziały sprawozdania) są uzupełniane wg wymagań w
opisie Projektu 2. i Harmonogramie Zajęć na WIKAMP KSR jako efekty zadań w~poszczególnych tygodniach}. 

\section{Cel}
Zwięzły (2-3 zdania) opis
problemu badawczego/obliczeniowego, uwzględniający część badawczą i implementacyjną. \\
\indent Zamieszczony opis (własny, nie skopiowany) zawiera
przypisy do literatury (bibliografii) zamieszczonej na końcu raportu/sprawozdania
zgodnie z~Polską Normą cytowania bibliografii (zob. materiały BG PŁ pt. ,,Bibliografia
załącznikowa''). Opis własny (niekopiowany) z przypisami do literatury -- minimum
teorii potrzebnej do zadania, konieczne ścisłe odniesienie elementów teorii do tego konkretnego zadania (zbiór
danych, liczba zmiennych i rekordów, itp.), tak by inżynier innej specjalności zrozumiał dalszy
opis tego konkretnego eksperymentu. \\
\noindent {\bf Sekcja uzupełniona jako efekt zadania Tydzień 08 wg Harmonogramu Zajęć
na WIKAMP KSR.}


\section{Charakterystyka podsumowywanej bazy danych}
Krótki opis bazy danych wybranej do podsumowywania, źródło, opis treści,
użyteczność/zastosowania. Liczba rekordów (min. $10\,000$ i koniecznie wszystkie tego
samego typu), liczba atrybutów możliwych do rozmycia (min. $10$), czyli o stosunkowo dużej
liczbie możliwych wartości. Zwyczajowe wartości lingwistyczne nadawane wybranym
atrybutom oraz dlaczego istnieje zwyczaj, zapotrzebowanie/inne powody
,,przekałdania'' tych danych na język
naturalny (a nie formalny) \cite{niewiadomski19, niewiadomski08}.\\
Realizacja bazy w wybranym DBMS. Rysunek lub tabela (fragment tabeli BD).\\ 
\noindent {\bf Sekcja uzupełniona jako efekt zadania Tydzień 08 wg Harmonogramu Zajęć
na WIKAMP KSR.}

\section{Atrybuty i liczności obiektów wyrażone zmiennymi lingwistycznymi}
Zmienne lingwistyczne dla wybranych 10 atrybutów z bazy danych, przedstawione w
formie wykresów funkcji przynależności i wzorów analitycznych, wymienione etykiety oraz objaśnione wszystkie
symbole ułatwiające czytelnikowi ich zrozumienie \cite{zadrozny06}. Zbędne jest
cytowanie definicji. Konieczne precyzyjnie podane przestrzenie rozważań każdej
zmiennej lingwistycznej, wzory i wykresy dla każdej wartości/etykiety.\\
Jw. kwantyfikatory lingwistyczne -- opisane etykietami, wykresami funkcji
przynależności i wzorami analitycznymi. Uzasadnione wiedzą dziedzinową wybrane
zakresy i etykiety. Precyzyjnie podane przestrzenie rozważań każdego kwantyfikatora 
lingwistycznego/rozmytego, wzory i wykresy dla każdej wartości/etykiety. Opisy własne z~przypisami do literatury, tak by inżynier innej specjalności zrozumiał dalszy
opis tego konkretnego ćwiczenia/eksperymentu. \\ 
\noindent {\bf Sekcja uzupełniona jako efekt zadania Tydzień 09 wg Harmonogramu Zajęć
na WIKAMP KSR.}

\section{Narzędzia obliczeniowe: projekt (wybór, implementacja) i diagram UML pakietu obliczeń rozmytych. Diagram UML generatora podsumowań}
\subsection{Diagram pakietu obliczeń rozmytych}
Diagram UML i zwięzły opis pakietu obliczeń rozmytych: źródło pakietu
(zewnętrzny/własny/hybrydowy), przypis do literatury. Krótka charakterystyka
najważniejszych klas i podstawowych dla zadania ich metod. \\
\noindent {\bf Sekcja uzupełniona jako efekt zadania Tydzień 10 wg Harmonogramu Zajęć
na WIKAMP KSR.}

\subsection{Diagram UML generatora podsumowań. Krótka instrukcja użytkownika} 
Diagram UML generatora podsumowań (warstwy obliczeniowej oraz interfejsu
użytkownika). Krótki ilustrowany opis jak użytkownik może korzystać z aplikacji, w~szczególności
wprowadzać parametry  podsumowań, odczytywać wyniki oraz definiować własne etykiety i
kwantyfikatory. Wersja JRE i inne wymogi niezbędne do uruchomienia aplikacji przez użytkownika na własnym komputerze. \\
\noindent {\bf Sekcja uzupełniona jako efekt zadania Tydzień 11 wg Harmonogramu Zajęć
na WIKAMP KSR.}

\section{ Jednopodmiotowe podsumowania lingwistyczne. Miary jakości, podsumowanie optymalne}
Wyniki kolejnych eksperymentów wg punktów 2.-4. opisu projektu 2.  Listy podsumowań
jednopodmiotowych i tabele/rankingi podsumowań dla danych atrybutów obowiązkowe i dokładnie opisane w ,,captions'' (tytułach), konieczny opis kolumn i wierszy tabel. Dla każdego podsumowania podane miary jakości oraz miara jakości podsumowania
optymalnego.\\
\noindent {\bf Sekcja uzupełniona jako efekt zadania Tydzień 11 wg Harmonogramu Zajęć
na WIKAMP KSR.}



\section{Wielopodmiotowe podsumowania lingwistyczne i~ich miary jakości} 
Wyniki kolejnych eksperymentów wg punktów 2.-4. opisu projektu 2. Uzasadnienie i
metoda podziału zbioru danych na rozłączne podmioty. Listy podsumowań
wielopodmiotowych i tabele/rankingi podsumowań dla danych atrybutów obowiązkowe i
dokładnie opisane w ,,captions'' (tytułach), konieczny opis kolumn i wierszy tabel.
Konieczne uwzględnienie wszystkich 4-ch form podsumowań wielopodmiotowych. 
\\ 

** Możliwe sformułowanie zagadnienia wielopodmiotowego podsumowania optymalnego **.\\

{**Ewentualne wyniki realizacji punktu ,,na ocenę 5.0'' wg opisu Projektu 2. i ich porównanie do wyników z
części obowiązkowej**.}\\

\noindent {\bf Sekcja uzupełniona jako efekt zadania Tydzień 12 wg Harmonogramu Zajęć
na WIKAMP KSR.}


\section{Dyskusja, wnioski}
Dokładne interpretacje uzyskanych wyników w zależności od parametrów klasyfikacji
opisanych w punktach 3.-4 opisu Projektu 2. 
Szczególnie istotne są wnioski o charakterze uniwersalnym, istotne dla podobnych zadań. 
Omówić i wyjaśnić napotkane problemy (jeśli były). Każdy wniosek/problem powinien mieć poparcie
w przeprowadzonych eksperymentach (odwołania do konkretnych wyników: tabel i miar
jakości). Ocena które wybrane kwantyfikatory, sumaryzatory, kwalifikatory i/lub ich
miary jakości mają małe albo duże znaczenie dla wiarygodności i jakości otrzymanych
agregacji/podsumowań.  \\
\underline{Dla końcowej oceny jest to najważniejsza sekcja} sprawozdania, gdyż prezentuje poziom
zrozumienia rozwiązywanego problemu.\\

** Możliwości kontynuacji prac w obszarze logiki rozmytej i wnioskowania rozmytego, zwłaszcza w kontekście pracy inżynierskiej,
magisterskiej, naukowej, itp. **\\

\noindent {\bf Sekcja uzupełniona jako efekt zadań Tydzień 11 i Tydzień 12 wg
Harmonogramu Zajęć na WIKAMP KSR.}


\section{Braki w realizacji projektu 2.}
Wymienić wg opisu Projektu 2. wszystkie niezrealizowane obowiązkowe elementy projektu, ewentualnie
podać merytoryczne (ale nie czasowe) przyczyny tych braków. 


\begin{thebibliography}{99}
 \bibitem{niewiadomski19} A. Niewiadomski, Zbiory rozmyte typu 2. Zastosowania w reprezentowaniu informacji.  Seria „Problemy współczesnej informatyki” pod redakcją L. Rutkowskiego. Akademicka Oficyna Wydawnicza EXIT, Warszawa, 2019.
\bibitem{zadrozny06} S. Zadrożny, Zapytania nieprecyzyjne i lingwistyczne podsumowania baz danych, EXIT, 2006, Warszawa
\bibitem{niewiadomski08} A. Niewiadomski, Methods for the Linguistic Summarization of Data: Applications of Fuzzy Sets and Their Extensions, Akademicka Oficyna Wydawnicza EXIT, Warszawa, 2008.
\end{thebibliography}

Literatura zawiera wyłącznie źródła recenzowane i/lub o potwierdzonej wiarygodności,
możliwe do weryfikacji i cytowane w sprawozdaniu. 
\end{document}
